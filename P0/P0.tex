
\documentclass{article}
\setlength{\parskip}{5pt} % esp. entre parrafos
\setlength{\parindent}{0pt} % esp. al inicio de un parrafo
\usepackage{amsmath} % mates
\usepackage[sort&compress,numbers]{natbib} % referencias
\usepackage{url} % que las URLs se vean lindos
\usepackage[top=25mm,left=20mm,right=20mm,bottom=25mm]{geometry} % margenes
\usepackage{hyperref} % ligas de URLs
\usepackage{graphicx} % poner figuras
\usepackage[spanish]{babel} % otros idiomas
\usepackage{listings}

\title{Tarea 0}
\author{Ismael Hernández}
\date{enero 19 2022}

\begin{document}

\maketitle

\section{Insertar ecuación e imagen}

Ejemplo Ecuación \eqref{equ1}:
\begin{equation}
  f(x) = 2 \sin(x) - \int_0^\infty \frac{1}{1 + x} \text{d}x.
  \label{equ1}
\end{equation}
Ejemplo \eqref{equ2}:
\begin{equation}
  f(x) = 2 \sin(x).
  \label{equ2}
\end{equation}

\begin{figure}[h] % figura
    \centering
    \includegraphics[width=30mm]{descarga.jpg} % archivo
    \caption{UANL tomado de \url{https://www.uanl.edu.mx} con licencia CC.}
    \label{uanl}
\end{figure}

\section{¿Cómo referenciar imagenes y fuentes?}
Ejemplo cita de fuentes \citep{ejemplo:1}.

Para hacer referencia a una figrua \ref{uanl}

\section{Insertar codigos .py }
\lstset{language=Python}
\lstset{frame=lines}
\lstset{caption={Codigo insertado directamente}}
\lstset{label={lst:code_direct}}
\lstset{basicstyle=\footnotesize}
\begin{lstlisting}
import numpy as np
from math import sqrt, exp, sin, cos, tan, tanh, log
import matplotlib.pyplot as plt 
import numpy as np
import pandas as pd
    



x=5
y=8
l=[x,y,x+x]

print('LA SUMA ES',sum(l))
data = [1, 4, 2, 4, 2, 5, 6, 7, 4, 76, 3, 2, 5, 6, 7]
from scipy.stats import describe
describe(data)

matriz=np.matrix([[1,3,4,5],[3,4,6,9]])
print(matriz)
f=matriz[1,3]

sin_f=sin(f)
if sin_f >0.1:
    print('mayor a 0.1')
    
datos = pd.read_csv('P0.csv', sep=' ')
print(datos)

plt.plot([1, 2, 3],[10,20,30]) 
plt.title('matplotlib.pyplot.plot() example 1') 
plt.draw() 
plt.show()
plt.close()

datos = [1.6, 4.6, 2.6, 3.6, 5.6, 6.6, 3.5, 2.2, 4.4, 5.2, 5.4, 7.6, 5.8, 4.4, 6.4]
plt.hist(datos)
plt.show()
plt.close()
\end{lstlisting}
\bibliography{simu}
\bibliographystyle{plainnat}





\end{document}
