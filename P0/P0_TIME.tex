\documentclass{article}
\usepackage[utf8]{inputenc}
\usepackage{listings}
\usepackage{pgfplots}
\pgfplotsset{compat = newest}
\usepackage{tikz}
  \usetikzlibrary{automata,topaths}
\usepackage[top=25mm,left=20mm,right=20mm,bottom=25mm]{geometry}
\title{P0 TIME}
\author{Ismael Crespo}
\date{February 2022}

\begin{document}

\maketitle

\section{Introducción}
Este reporte presenta los resultados de la medición de tiempo y memoria requerida para trabajar con matrices y vectores de \emph{n filas y n columnas} en R y Python. Los valores de \emph{n} fueron seleccionados haciendo pruebas para determinar el máximo permitido por el equipo de computo utilizado. 

\section{Medición de tiempos y espacio }
El equipo utilizado para la medición de rendimientos es una computadora portatil \emph{Acer E5-573 } con un Procesador \emph{Intel Core i5 2.20GHz},una memoria RAM instalada de 8.0 GB y un sistema operativo de 64 bits.
\subsection{Medición en R.} 
Se crearon matrices de $2^n * 2^n$ con números aleataorios y se utilizaron las funciones sysetm.time y object.size Listing \ref{lst1} para para medir el tiempo requerido para crear cada matriz y el espacio requerido en la memoria RAM. Los intervalos de valor n se encontraron experimentando uno a uno hasta encontrar un máximo permisible por el equipo de computo en en n=14. Los resultados se presentan en la tabla \ref{tabla 1} \ref{grafica1}

\begin{table}[h]
\centering
\begin{tabular}{|l|l|l|l|}
\hline
n  & $2^n$ & Espacio en RAM (Bytes) & Tiempo (Seg) \\ \hline
2  & 4                    & 344                    & 0            \\ \hline
3  & 8                    & 728                    & 0            \\ \hline
4  & 16                   & 2264                   & 0            \\ \hline
5  & 32                   & 8408                   & 0            \\ \hline
6  & 64                   & 32984                  & 0            \\ \hline
7  & 128                  & 131288                 & 0            \\ \hline
8  & 256                  & 524504                 & 0            \\ \hline
9  & 512                  & 2097368                & 0.01         \\ \hline
10 & 1024                 & 8388824                & 0.04         \\ \hline
11 & 2048                 & 33554648               & 0.17         \\ \hline
12 & 4096                 & 134217944              & 0.79         \\ \hline
13 & 8192                 & 536871128              & 2.82         \\ \hline
14 & 16384                & 2147483864             & 17.91        \\ \hline
\end{tabular}
\caption{Tiempo y espacio requerido para cada matriz}
\label{tabla 1}
\end{table}
\begin{figure}
    \centering 
   
    \caption{Tiempo requerido para ordenar una matriz de $2^n * 2^n$ elementos  aleatorios en R}
    \label{grafica1}

\begin{tikzpicture}
\begin{axis}[
    xmin = 2, xmax = 14,
    ymin = 0, ymax = 19,
    xtick distance = 1,
    ytick distance = 1,
    grid = both,
    minor tick num = 1,
    major grid style = {lightgray},
    minor grid style = {lightgray!25},
    width = \textwidth,
    height = 0.5\textwidth,
    xlabel = {$Tiempo$},
    ylabel = {$n$},]
 
% Plot data from a file
\addplot[
    smooth,
    thin,
    red,
    dashed
] file[skip first] {R_time.dat};


\end{axis}
 
\end{tikzpicture}
\end{figure}
\lstset{language=Python}
\lstset{frame=lines}
\lstset{caption={Codigo en R para medir rendimeintos}}
\lstset{label={lst1}}
\lstset{basicstyle=\footnotesize}
\begin{lstlisting}
for(n in 2:14) {k=2^n; cat(k,system.time(matrix(runif(k*k),
nrow=k))[3], '\n')}
library(pryr)
for(n in 2:14) {k=2^n; cat(k,object_size(matrix(runif(k*k),
nrow=k)), '\n')}
\end{lstlisting}


\subsection{Medición en Python.} 
Para medir el rendimiento en Python sin necesidad de usar una libreria adicional se utilizó el Listing \ref{lst2},y en la grafica \ref{grafica2} que guarda la hora antes de comenzar a ordenar número aleatorios en vectores de $2^n$, para restarsela a la hora final y obtener el tiempo que tarda en hacer estos procedimeinto, el espacio ocupado por cada vector se obtuvo utilizando la función \emph{getsizeof}. Los resultados se presentan en la tabla \ref{tabla 2}.
\lstset{language=Python}
\lstset{frame=lines}
\lstset{caption={Codigo en Python para medir rendimeintos}}
\lstset{label={lst2}}
\lstset{basicstyle=\footnotesize}
\begin{lstlisting}
from random import randint
desde = 1
hasta = 1000
menor = 15
mayor = 21
from time import time # traer libreria
for k in range(menor, mayor + 1):
    n = 2 ** k
    lista = [ randint(desde, hasta) for i in range(n) ]
    antes = time() # ver la hora
    lista.sort()
    despues = time() # volver ver la hora
    diferencia = despues - antes # segundos
    print('ordenar', n, 'elementos toma', diferencia, 'segundos')
    
import sys
from sys import getsizeof
import numpy as np
from random import randint
menor=15
mayor=21
for k in range(menor, mayor + 1):
    n=2**k
    lista = [ randint(1, 1000) for i in range(n) ]
    espacio=getsizeof(lista)
    print('un vector de',n,'elementos ocupa',espacio,'bytes de memoria')
\end{lstlisting}
\begin{table}[h]
\centering
\begin{tabular}{|l|l|l|l|}
\hline
n  & $2^n$& Tiempo [Seg] & Espacio [Bytes]\\ \hline
15  & 32768                    & 0.0      &  277336                    \\ \hline
16  & 65536                    & 0.011695 & 562488                           \\ \hline
17  & 131072                   & 0.0325872  & 1140568                        \\ \hline
18  & 262144                   & 0.0680267   &  2312472                        \\ \hline
19  & 524288                   & 0.14938402   & 4688312                           \\ \hline
20  & 1048576                  & 0.22246694  & 8448728                           \\ \hline
21  & 2097152                  & 0.41567270  & 17128280                           \\ \hline
\end{tabular}
\caption{Tiempo requerido para ordendar cada vector en Python}
\label{tabla 2}
\end{table}

\begin{figure}
    \centering 
   
    \caption{Tiempo requerido para ordenar un vector de $2^n$ números aleatorios en Python}
    \label{grafica2}

\begin{tikzpicture}
\begin{axis}[
    xmin = 15, xmax = 21,
    ymin = 0, ymax = .5,
    xtick distance = 1,
    ytick distance = .1,
    grid = both,
    minor tick num = 1,
    major grid style = {lightgray},
    minor grid style = {lightgray!25},
    width = \textwidth,
    height = 0.5\textwidth,
    xlabel = {$Tiempo$},
    ylabel = {$n$},]
 
% Plot data from a file
\addplot[
    smooth,
    thin,
    red,
    dashed
] file[skip first] {python_time.dat};


\end{axis}
 
\end{tikzpicture}
\end{figure}

\section{Conclusiones}
El equipo de computo utilizado presenta un numero maximo permitible para realizar trabajos matriciales y dependen de cada herramienta utilizada para progrmar así como de las tareas realizadas en paralelo. Es necesario conocer los estados optimos para sacar el máximo potencial al equipo.
\end{document}
